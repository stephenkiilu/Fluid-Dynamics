%%%%%%%%%%%%%%%%%%%%%%%%%%%%%%%%%%%%%%%%%%%%%%%%%%%%%%%%%%%%%%%%%%%%%%%%%%%%%%%%
%%%%%%%%%%%%%%%%%%%%%%%%%%%%%%%%%%%%%%%%%%%%%%%%%%%%%%%%%%%%%%%%%%%%%%%%%%%%%%%%
%%% Template for AIMS Rwanda Assignments         %%%              %%%
%%% Author:   AIMS Rwanda tutors                             %%%   ###        %%%
%%% Email: tutors2017-18@aims.ac.rw                               %%%   ###        %%%
%%% Copyright: This template was designed to be used for    %%% #######      %%%
%%% the assignments at AIMS Rwanda during the academic year %%%   ###        %%%
%%% 2017-2018.                                              %%%   #########  %%%
%%% You are free to alter any part of this document for     %%%   ###   ###  %%%
%%% yourself and for distribution.                          %%%   ###   ###  %%%
%%%                                                         %%%              %%%
%%%%%%%%%%%%%%%%%%%%%%%%%%%%%%%%%%%%%%%%%%%%%%%%%%%%%%%%%%%%%%%%%%%%%%%%%%%%%%%%
%%%%%%%%%%%%%%%%%%%%%%%%%%%%%%%%%%%%%%%%%%%%%%%%%%%%%%%%%%%%%%%%%%%%%%%%%%%%%%%%


%%%%%% Ensure that you do not write the questions before each of the solutions because it is not necessary. %%%%%% 

\documentclass[12pt,a4paper]{article}

%%%%%%%%%%%%%%%%%%%%%%%%% packages %%%%%%%%%%%%%%%%%%%%%%%%
\usepackage{amsmath}
\usepackage{amssymb}
\usepackage[utf8]{inputenc}
\usepackage[english]{babel}
\usepackage{hyperref}
\usepackage{amsthm}
\usepackage{amsfonts}
\usepackage{graphicx}
\usepackage[all]{xy}
\usepackage{tikz}
\usepackage{url}
\usepackage{placeins}
\usepackage{float}
\usepackage{verbatim}
\usepackage[left=2cm,right=2cm,top=3cm,bottom=2.5cm]{geometry}
\usepackage{hyperref}
\usepackage{caption}
\usepackage{subcaption}
\usepackage{psfrag}

%%%%%%%%%%%%%%%%%%%%% students data %%%%%%%%%%%%%%%%%%%%%%%%
\newcommand{\student}{Stephen Kiilu}
\newcommand{\course}{Fluid Dynamics}
\newcommand{\assignment}{FD2}

%%%%%%%%%%%%%%%%%%% using theorem style %%%%%%%%%%%%%%%%%%%%
\newtheorem{thm}{Theorem}
\newtheorem{lem}[thm]{Lemma}
\newtheorem{defn}[thm]{Definition}
\newtheorem{exa}[thm]{Example}
\newtheorem{rem}[thm]{Remark}
\newtheorem{coro}[thm]{Corollary}
\newtheorem{quest}{Question}[section]

%%%%%%%%%%%%%%  Shortcut for usual set of numbers  %%%%%%%%%%%

\newcommand{\N}{\mathbb{N}}
\newcommand{\Z}{\mathbb{Z}}
\newcommand{\Q}{\mathbb{Q}}
\newcommand{\R}{\mathbb{R}}
\newcommand{\C}{\mathbb{C}}

%%%%%%%%%%%%%%%%%%%%%%%%%%%%%%%%%%%%%%%%%%%%%%%%%%%%%%%555
\begin{document}

%%%%%%%%%%%%%%%%%%%%%%% title page %%%%%%%%%%%%%%%%%%%%%%%%%%
\thispagestyle{empty}
\begin{center}
\textbf{AFRICAN INSTITUTE FOR MATHEMATICAL SCIENCES \\[0.5cm]
(AIMS RWANDA, KIGALI)}
\vspace{1.0cm}
\end{center}

%%%%%%%%%%%%%%%%%%%%% assignment information %%%%%%%%%%%%%%%%
\noindent
\rule{17cm}{0.2cm}\\[0.3cm]
Name: \student \hfill Assignment Number: \assignment\\[0.1cm]
Course: \course \hfill Date: \today\\
\rule{17cm}{0.05cm}
\vspace{1.0cm}

\section*{Question 1}
We are given the following dynamical systems;
\begin{enumerate}
\item[(1)]
\begin{eqnarray*}
\frac{dx}{dt}&=&\begin{pmatrix}
5 & 1 \\
3& 1 
\end{pmatrix}x
\end{eqnarray*}
We calculate eigenvalues and the corresponding eigenvalues.
The eigenvalues are ;
\begin{eqnarray*}
\lambda_1=3+\sqrt{7}\\
\lambda_2=3-\sqrt{7}\\
\end{eqnarray*}
And corresponding eigenvectors are
\begin{eqnarray*}
\textbf{v}_1=\begin{pmatrix}
2+ \sqrt{7}\\
3 
\end{pmatrix}\\
\textbf{v}_2=\begin{pmatrix}
- \sqrt{7}+2\\
 3
\end{pmatrix}
\end{eqnarray*}
The general solution becomes
\begin{eqnarray*}
\begin{pmatrix}
x(t) \\
y(t)
\end{pmatrix}=c_1 \begin{pmatrix}
2+ \sqrt{7}\\
3 
\end{pmatrix}e^{(3+\sqrt{7})t}+c_2 \begin{pmatrix}
 - \sqrt{7}+2\\
 3
\end{pmatrix}e^{(3-\sqrt{7})t}\\
\end{eqnarray*}
The following figure \ref{fig 1} shows the corresponding phase portrait. The eigenvalues are all positive and real. We have \textbf{improper node} and it is \textbf{unstable}.
\begin{figure}[H]
\includegraphics[width=14cm]{a}
\centering
\caption{improper node}
\label{fig 1}
\end{figure}
\item[(2)]
\begin{eqnarray*}
\frac{dx}{dt}&=&\begin{pmatrix}
2 & -5 \\
1& -2
\end{pmatrix}x
\end{eqnarray*}


We calculate eigenvalues and the corresponding eigenvalues.
The eigenvalues are ;

\begin{eqnarray*}
\lambda_1=i\\
\lambda_2=-i\\
\end{eqnarray*}

And corresponding eigenvectors are
\begin{eqnarray*}
\textbf{v}_1=\begin{pmatrix}
2+i\\
1 
\end{pmatrix}\\
\textbf{v}_2=\begin{pmatrix}
2-i\\
 1
\end{pmatrix}
\end{eqnarray*}


The general solution becomes
\begin{eqnarray*}
\begin{pmatrix}
x(t) \\
y(t)
\end{pmatrix}=c_1 \begin{pmatrix}
\cos2t-2 \sin 2t\\
\cos 2t
\end{pmatrix}+c_2 \begin{pmatrix}
\sin 2t+2 \cos 2t\\
 \sin 2t
\end{pmatrix}\\
\end{eqnarray*}

The following figure \ref{fig 2} shows the phase portrait. We have complex eigenvalues and the real part is 0. This means that we have \textbf{center} and it is \textbf{stable.}

\begin{figure}[H]
\includegraphics[width=14cm]{b}
\centering
\caption{center}
\label{fig 2}
\end{figure}
\item[(3)]
\begin{eqnarray*}
\frac{dx}{dt}&=&\begin{pmatrix}
1 & -5 \\
1& -3
\end{pmatrix}x
\end{eqnarray*}




We calculate eigenvalues and the corresponding eigenvalues.
The eigenvalues are ;

\begin{eqnarray*}
\lambda_1=-1+i\\
\lambda_2=-1-i\\
\end{eqnarray*}

And corresponding eigenvectors are
\begin{eqnarray*}
\textbf{v}_1=\begin{pmatrix}
2+i\\
1 
\end{pmatrix}\\
\textbf{v}_2=\begin{pmatrix}
2-i\\
 1
\end{pmatrix}
\end{eqnarray*}


The general solution becomes
\begin{eqnarray*}
\begin{pmatrix}
x(t) \\
y(t)
\end{pmatrix}=c_1 e^{-t}\left(  \begin{pmatrix}
2\\
1
\end{pmatrix} \cos t -\begin{pmatrix}
1\\
0
\end{pmatrix} \sin t\right)+c_2 e^{-t}\left(  \begin{pmatrix}
2\\
1
\end{pmatrix} \sin t -\begin{pmatrix}
1\\
0
\end{pmatrix} \cos t\right)
\end{eqnarray*}

The following figure \ref{fig 3} shows the corresponding phase portrait. We have complex eigenvalues and the real part is negative. This means that we a \textbf{stable spiral} and it is \textbf{asymptotically stable.}

\begin{figure}[H]
\includegraphics[width=14cm]{c}
\centering
\caption{stable spiral}
\label{fig 3}
\end{figure}



\item[(4)]
\begin{eqnarray*}
\frac{dx}{dt}&=&\begin{pmatrix}
2 & -1 \\
3& -2
\end{pmatrix}x
\end{eqnarray*}


We calculate eigenvalues and the corresponding eigenvalues.
The eigenvalues are ;

\begin{eqnarray*}
\lambda_1=1\\
\lambda_2=-1\\
\end{eqnarray*}

And corresponding eigenvectors are
\begin{eqnarray*}
\textbf{v}_1=\begin{pmatrix}
1\\
1 
\end{pmatrix}\\
\textbf{v}_2=\begin{pmatrix}
1\\
 3
\end{pmatrix}
\end{eqnarray*}


The general solution becomes
\begin{eqnarray*}
\begin{pmatrix}
x(t) \\
y(t)
\end{pmatrix}=c_1 \begin{pmatrix}
1\\
1
\end{pmatrix}e^t+c_2  \begin{pmatrix}
1\\
3
\end{pmatrix}e^{-t}\\
\end{eqnarray*}

The following figure \ref{fig 4} shows the corresponding phase portrait. The eigenvalues are all real and of opposite sign, that is , we have a negative and positive eigenvalues. This implies that we have a \textbf{saddle point} and it is \textbf{unstable.}

\begin{figure}[H]
\includegraphics[width=14cm]{d}
\centering
\caption{saddle point}
\label{fig 4}
\end{figure}


\end{enumerate}
\section*{Question 2}
We are given the following dynamical system;
\begin{eqnarray*}
\frac{dx}{dt}&=&y-x^2\\
\frac{dy}{dt}&=&x-2\\
\end{eqnarray*}
\begin{enumerate}
\item[(1)]
Calculate the equilibrium point. We form to equations and solve to obtain equilibrium points
\begin{eqnarray*}
y&=&x^2\\
x&=&2\\
\end{eqnarray*}
Solving we obtain;
\begin{eqnarray*}\
x=2 \,\, \text{and};y=4\
\end{eqnarray*}
Our equilibrium points becomes $(2,4)$.
\item[(2)]
Linearize the dynamical system around the equilibrium points. We first find the Jacobian, J of our system . Let;

\begin{eqnarray*}
f_1&=&y-x^2\\
f_2&=&x-2\\
 J&=&
\begin{pmatrix} 
\frac{\partial f_1 }{\partial x} & \frac{\partial f_1 }{\partial y} \\
\frac{\partial f_2 }{\partial x} & \frac{\partial f_2 }{\partial y}
\end{pmatrix}
\end{eqnarray*}
We apply Jacobian transformation to our system and obtain;
\begin{eqnarray*}
 A&=&
\begin{pmatrix} 
-2x & 1 \\
1 & 0
\end{pmatrix}
\end{eqnarray*}
But $x=2$, we obtain A, as;
\begin{eqnarray*}
 A&=&
\begin{pmatrix} 
-4 & 1 \\
1 & 0
\end{pmatrix}
\end{eqnarray*}
\item[(3)]
We now obtain characteristic equation and solve to obtain the eigenvalues;
\begin{eqnarray*}
\det (A-\lambda \mathbb{I})&=&0\\
\lambda ^2+4\lambda-1 &=&0\\
\text{we solve to obtain the eigenvalues as}\\
\lambda_1=-2+\sqrt{5}\\
\lambda_2=-2-\sqrt{5}
\end{eqnarray*}
We solve to obtain the corresponding eigenvectors
\begin{eqnarray*}
\text{for $\lambda_1=-2+\sqrt{5}$}\\ \\
\begin{pmatrix} 
-2-\sqrt{5} & 1 \\
1 & 2-\sqrt{5}\\
\end{pmatrix}
\begin{pmatrix} 
 x \\
y\\
\end{pmatrix}=
 0 \\
0\\
(-2-\sqrt{5})x +y=0\\
x+(2-\sqrt{5})y =0\\ \\
\text{solving ;}\\
x=(-2+\sqrt{5})y\\
\text{let } y=1\\
x=-2+\sqrt{5}
\end{eqnarray*}
Our eigenvector becomes;
\begin{eqnarray*}
\textbf{v}_1 &=&\begin{pmatrix} 
 -2+\sqrt{5}\\
1\\
\end{pmatrix}
\end{eqnarray*}
Similarly we find the other eigenvector;
\begin{eqnarray*}
\text{for $\lambda_2=-2-\sqrt{5}$}\\ \\
\begin{pmatrix} 
-2+\sqrt{5} & 1 \\
1 & 2+\sqrt{5}\\
\end{pmatrix}
\begin{pmatrix} 
 x \\
y\\
\end{pmatrix}=
 0 \\
0\\
(-2+\sqrt{5})x +y=0\\
x+(2+\sqrt{5})y =0\\ \\
\text{solving ;}\\
x=(-2-\sqrt{5})y\\
\text{let } y=1\\
x=-2-\sqrt{5}
\end{eqnarray*}
Our eigenvector becomes;
\begin{eqnarray*}
\textbf{v}_2 &=&\begin{pmatrix} 
 -2-\sqrt{5}\\
1\\
\end{pmatrix}
\end{eqnarray*}
Having solved the eigenvalues and the corresponding eigenvectors, we notice that, the eigenvalues are real and of opposite sign. This implies that we have \textbf{saddle point} and it is \textbf{unstable}.
We can now write the general solution as;
\begin{eqnarray*}
\begin{pmatrix}
x(t) \\
y(t)
\end{pmatrix}=c_1 
 \begin{pmatrix} 
 -2+\sqrt{5}\\
1\\

\end{pmatrix}e^{(-2+\sqrt{5})t}+c_2 \begin{pmatrix} 
 -2-\sqrt{5}\\
1\\
\end{pmatrix}e^{(2-\sqrt{5})t}\\
\end{eqnarray*}
\end{enumerate}

The following figure \ref{fig 5} shows the corresponding phase portrait.

\begin{figure}[H]
\includegraphics[width=14cm]{e}
\centering
\caption{saddle point}
\label{fig 5}
\end{figure}

\newpage
\begin{thebibliography}{99}

  \bibitem{notes} Prof.Yoshifumi Kimura  ,{\em Lecture notes,}  2021.\\
  \bibitem{url} \url{https://courses.lumenlearning.com/boundless-physics/chapter/damped-and-driven-oscillations/} 


\end{thebibliography}

\end{document}